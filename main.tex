\documentclass{article}
\usepackage[utf8]{inputenc}
\usepackage[spanish]{babel}
\usepackage{listings}
\usepackage{graphicx}
\graphicspath{ {images/} }
\usepackage{cite}

\begin{document}

\begin{titlepage}
    \begin{center}
        \vspace*{1cm}
            
        \Huge
        \textbf{Ideación}
            
        \vspace{0.5cm}
        \LARGE
        Actividad 1 Proyecto Final
            
        \vspace{1.5cm}
            
        \textbf{Mariana Noreña Vásquez}
        
        \textbf{Santiago Vélez Arboleda}
        
            
        \vfill
            
        \vspace{0.8cm}
            
        \Large
        Despartamento de Ingeniería Electrónica y Telecomunicaciones\\
        Universidad de Antioquia\\
        Medellín\\
        Marzo de 2021
            
    \end{center}
\end{titlepage}

\tableofcontents
\newpage
\section{Introducción}\label{intro}
En este documento se plasmarán las ideas principales que abarca el proyecto final del curso. Contendrá el contexto,en dónde se va a desarrollar, tipo de juego, ambientación, mecánicas, entre otros.  

\section{Sección de contenido} \label{contenido}
A continuación se presentará la metodología que tendrá el juego:
\subsection{Contexto.}
Empezando por el nombre del videojuego, el cuál será: GEHRIN. Tendrá como objetivo que el jugador desarrolle nuevas mecánicas lógico motrices para sobrepasar obstáculos y enemigos que se presentarán a lo largo de los niveles.  

\subsection{Desarrollo}
%
El IDE de desarrollo de GEHRIN será QT Creator. El videojuego se dividirá en pequeños problemas que van a estar representados en funciones específicas que en conjunto harán que el juego compile. 

\subsection{Tipo de juego}
Será un juego de plataformas en 2D. El cual contendrá obstáculos, objetos (como armas, herramientas, etc) y monedas que el usuario recolectará en el transcurso del juego.  

\subsection{Ambientación}
El jugador podrá escoger entre tres avatares los cuales están inspirados en series y videojuegos de la infancia. Mínimo se tendrán tres niveles los cuales serán de diferente ambientación de los personajes propuestos. Por ejemplo, si el personaje seleccionado proviene de la serie KND los chicos del barrio, los escenarios del videojuego serán paralelos a la ambientación mostrada en la serie. Sonoramente, los escenarios tendrán una ambientación semejante a la serie o videojuego que representa al mismo. Ejemplificando, si el escenario está ambientado en Super Mario la banda sonora de fondo también lo será.
\subsection{Mecánicas}
El jugador deberá eliminar oponentes mientras esquiva obstáculos moviéndose a lo largo del escenario. Podrá disparar para infligir daño al enemigo y saltar para sobrepasar los obstáculos. 
De igual forma, deberá de desarrollar una habilidad lógico motriz para avanzar en los niveles del juego. 

\bibliographystyle{IEEEtran}
\bibliography{references}

\end{document}
